%Input
Let \( x \) be the original input data point, \( y \) be the true label, and \( A \) be a set of attributes relevant to the classification task. The MultAV Adversarial Attack aims to generate adversarial examples by manipulating multiple attributes simultaneously.

%Output
The output of the MultAV Adversarial Attack is a perturbed input \( \tilde{x} \) that is designed to mislead the model while altering multiple attributes.

%Formula
The MultAV Adversarial Attack can be formulated as follows:
1. Initialize the perturbation \( \delta \):
   $
   \tilde{x} = x + \delta,
   $
   where \( \|\delta\| \leq \epsilon \).
2. Define the objective function to minimize the loss for a target class \( c \):
   $
   \text{minimize } L(f(\tilde{x}), c),
   $
   where \( f \) is the model and \( L \) is the loss function.
3. Use a variational approach to model the perturbation across multiple attributes:
   $
   \delta^{(t+1)} = \delta^{(t)} + \alpha \cdot \sum_{a \in A} \nabla_{\delta} L(f(x + \delta^{(t)}), a),
   $
   where \( \alpha \) is the step size and the summation is over all attributes \( A \).
4. Update the perturbed input:
   $
   \tilde{x} = x + \delta^{(t+1)}.
   $

%Explanation
The MultAV (Mult-Attribute Variational) Adversarial Attack generates adversarial examples by strategically manipulating multiple relevant attributes of the input \( x \). By optimizing the perturbation \( \delta \) to minimize the loss associated with a target class \( c \), the attack effectively leverages variations across different attributes to create a robust adversarial input \( \tilde{x} \). This method highlights the vulnerability of machine learning models to adversarial examples that alter several features at once, potentially leading to misclassification while maintaining a plausible appearance.
