%Input
Let \( x \) be the original input and \( f_\theta \) be a classifier with parameters \( \theta \). The goal is to find an adversarial example \( x^* \) such that \( f_\theta(x^*) \neq y \), where \( y \) is the true label of \( x \).

%Output
The perturbed input \( x^* \) satisfying the adversarial condition:
\[
x^* = x + \delta
\]
where the perturbation \( \delta \) is determined as:
\[
\delta = \arg\max_{\|\delta\|_p \leq \epsilon} \left( L(f_\theta(x + \delta), y) \right)
\]
subject to:
\[
f_\theta(x^*) \neq y.
\]

%Formula
$x^* = x + \delta$
where the perturbation $\delta$ is determined by:
$\delta = \arg\max_{\|\delta\|_p \leq \epsilon} \left( L(f_\theta(x + \delta), y) \right)$
subject to the constraint that:
$f_\theta(x^*) \neq y$
where:
- $L$ is the loss function that measures the difference between the model's prediction and the true label.

%Explanation
The One-Shot Constrained Norm Attack (OS-CNA) is a variant of the original method. The key difference lies in incorporating a new constraint, $\|\delta\|_p \leq \epsilon$, which limits the magnitude of the perturbation $\delta$. This modification makes the attack more robust and stealthy by reducing the likelihood of detection.