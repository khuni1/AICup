%Input  
$\mathcal{D}_{\text{train}}$: Original training dataset.  
$\mathcal{D}_{\text{poison}}$: Poisoned data to be inserted.  
$f$: Target machine learning model.  
$\mathcal{L}(f, \mathcal{D}_{\text{train}})$: Loss function of the target model.  
$\epsilon$: Perturbation constraint to ensure imperceptibility.  

%Output  
$\mathcal{D}_{\text{train}}^{\text{poisoned}}$: The modified training dataset such that:  
\[
f(\mathcal{D}_{\text{train}}^{\text{poisoned}}) \neq f(\mathcal{D}_{\text{train}})
\]

%Formula
1. Construct Poisoned Instances:  
   Create $\mathcal{D}_{\text{poison}}$ such that its samples satisfy a specific target objective, e.g., misclassification, by manipulating input features:  
   \[
   x_{\text{poison}} = x_{\text{original}} + \delta, \quad \|\delta\| \leq \epsilon
   \]  

2. Poison Data Insertion:  
   Combine the poisoned data with the original training set:  
   \[
   \mathcal{D}_{\text{train}}^{\text{poisoned}} = \mathcal{D}_{\text{train}} \cup \mathcal{D}_{\text{poison}}
   \]

3. Retrain the Model:  
   Optimize the target model on the poisoned dataset:  
   \[
   f_{\text{poisoned}} = \arg \min_f \mathcal{L}(f, \mathcal{D}_{\text{train}}^{\text{poisoned}})
   \]  

%Explanation
The Poison Insertion Attack is a data poisoning method where carefully crafted poisoned samples are added to the training dataset. The primary goals include:  

1. Objective: Induce specific model behavior, such as misclassification or incorrect predictions, by retraining the model on the modified dataset.  

2. Perturbation Control: The added poison samples are constrained by a perturbation limit $\epsilon$ to ensure that they are indistinguishable from legitimate samples and evade detection.  

3. Impact: Poisoning can degrade model performance or achieve targeted manipulation, depending on the crafted poison samples.  

4. Practical Considerations: Requires access to the training pipeline and knowledge of the model’s dataset to inject poisons effectively.
