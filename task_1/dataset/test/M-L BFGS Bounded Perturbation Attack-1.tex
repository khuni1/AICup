%Input
The input consists of:
- Initial input sample $x^{(0)}$
- Learning rate sequence $\alpha_k$
- Approximate Hessian inverse $H_k$
- Gradient $\nabla f(x^{(k)})$
- Perturbation constraint $\epsilon$
- Projection function $h_k(z)$

%Output
The output is the adversarial example $x^*$, which satisfies the perturbation constraint while maximizing the attack effectiveness:

\[
x^* = \arg\max_{x \in S} f(x)
\]

where $S$ is the constrained search space defined by the bounded perturbation strategy.

%Formula
$x^{(k+1)} = x^{(k)} - \alpha_k H_k \nabla f(x^{(k)})$
$z_{k,i}(\epsilon) = \max_{z_{k-1}(\epsilon) \leq z_{k-1} \leq z_{k-1}(\epsilon)} e^T_i h_k(z_{k-1})$

%Explanation
This variant L-BFGS Bounded Perturbation Attack is derived from the BFGS optimization algorithm with limited-memory version. It introduces a bounded perturbation strategy that incorporates an additional constraint to ensure the attacks remain within a specific range of perturbations. This approach maintains the core principle of the original attack while improving its stealthiness and targetedness.

The key difference between this variant and the main perturbation core is the introduction of a bounded perturbation strategy, which ensures that the attacks remain within a specific range of perturbations. This constraint is represented by the formula $z_{k,i}(\epsilon) = \max_{z_{k-1}(\epsilon) \leq z_{k-1} \leq z_{k-1}(\epsilon)} e^T_i h_k(z_{k-1})$. This additional constraint makes the attack variant more targeted and stealthy, while maintaining its effectiveness.