%Input
$X = \{\mathbf{x}_1, \mathbf{x}_2, \ldots, \mathbf{x}_n\} \subset \mathbb{R}^d$: Original dataset.
$m$: Number of poisoned points to add.
Distance metric $d(\mathbf{x}_i, \mathbf{x}_j)$: Typically Euclidean distance.
Clustering function $C$: Complete-linkage clustering algorithm.

%Output
$X_{\text{poisoned}}$: Modified dataset including poisoned points $\{\mathbf{z}_1, \mathbf{z}_2, \ldots, \mathbf{z}_m\}$.

%Formula
1.Objective:  
   Maximize the disruption in the clustering process by solving:
   \[
   \max_{\{\mathbf{z}_i\}_{i=1}^m} \quad D(C(X_{\text{poisoned}}), C(X)),
   \]
   where $D(\cdot, \cdot)$ measures the distance between the dendrograms of the poisoned dataset $X_{\text{poisoned}}$ and the original dataset $X$.

2.Perturbation Strategy:  
   Poisoned points $\{\mathbf{z}_i\}_{i=1}^m$ are crafted to manipulate the maximum inter-cluster distances:
   \[
   \mathbf{z}_i = \arg \max_{\mathbf{z}} \left[ \max_{\mathbf{x} \in X} d(\mathbf{z}, \mathbf{x}) \right],
   \]
   ensuring they create large linkage distances to delay or disrupt desired merges.

%Explanation
1.Complete-Linkage Clustering:
   - In complete-linkage clustering, the distance between two clusters is defined as:
     \[
     d(C_i, C_j) = \max_{\mathbf{x} \in C_i, \mathbf{y} \in C_j} d(\mathbf{x}, \mathbf{y}).
     \]
   - Clusters are iteratively merged based on the largest inter-cluster distances.

2.Attack Strategy:
   - Poisoned points $\{\mathbf{z}_i\}_{i=1}^m$ are designed to manipulate the largest distances between clusters.
   - This delays certain cluster merges or forces undesired merges, disrupting the clustering hierarchy.

3.Optimization:
   - For each poisoned point $\mathbf{z}_i$, solve:
     \[
     \mathbf{z}_i = \arg \max_{\mathbf{z}} \left[ \max_{\mathbf{x} \in X} d(\mathbf{z}, \mathbf{x}) \right].
     \]
   - This ensures the poisoned points affect the maximum distance metric in complete-linkage clustering.

4.Output:
   - The poisoned dataset is:
     \[
     X_{\text{poisoned}} = X \cup \{\mathbf{z}_1, \mathbf{z}_2, \ldots, \mathbf{z}_m\}.
     \]

Key Characteristics
- Target: Hierarchical clustering with complete-linkage.
- Effect: Alters the dendrogram and final cluster assignments by exploiting the sensitivity of complete-linkage to maximum distances.
- Metric: Perturbations are evaluated by the difference $D(C(X_{\text{poisoned}}), C(X))$, which quantifies the disruption in clustering structure.