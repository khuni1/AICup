%Input
Let \( x \) be the original input data, \( y \) be the true label, and \( f_{\theta} \) be the target model. The Morpheus attack aims to generate adversarial examples by applying a series of transformations to the input, specifically targeting the model's vulnerabilities.

%Output
The output of the Morpheus attack is a modified input \( x^* \) that misleads the model while preserving the semantic content of the original input.

%Formula
The Morpheus adversarial attack can be formulated as follows:
1. Initialize the input and true label:
   $
   (x, y).
   $
2. Define a set of transformation functions \( \mathcal{T} \) that can alter the input, such as noise addition, scaling, or rotation:
   $
   \mathcal{T} = \{t_1, t_2, \ldots, t_n\}.
   $
3. For each transformation \( t_i \in \mathcal{T} \):
   - Apply the transformation to the input:
   $
   x' = t_i(x).
   $
   - Evaluate the model's prediction:
   $
   \hat{y} = f_{\theta}(x').
   $
   - If \( \hat{y} \neq y \), then \( x' \) is a candidate adversarial example.
4. The objective is to find a sequence of transformations that produces:
   $
   x^* = t_k(t_{k-1}(\ldots t_1(x) \ldots)),
   $
   while ensuring:
   $
   f_{\theta}(x^*) \neq y.
   $

%Explanation
The Morpheus attack generates adversarial examples by applying a series of transformations to the original input \( x \). By exploring different combinations of transformations, the attack aims to produce a new input \( x^* \) that remains similar to the original while successfully misleading the model into making an incorrect classification. This method highlights the adaptive nature of adversarial attacks and the necessity for robust defenses against various input manipulations.
