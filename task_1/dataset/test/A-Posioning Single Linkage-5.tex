%Input
$X = \{\mathbf{x}_1, \mathbf{x}_2, \ldots, \mathbf{x}_n\} \subset \mathbb{R}^d$: Original dataset.
$m$: Number of poisoned points to add.
Distance metric $d(\mathbf{x}_i, \mathbf{x}_j)$: Typically Euclidean distance.
Clustering function $C$: Single-linkage clustering algorithm.

%Output
$X_{\text{poisoned}}$: Modified dataset including poisoned points $\{\mathbf{z}_1, \mathbf{z}_2, \ldots, \mathbf{z}_m\}$.

%Formula
1.Objective: Maximize the disruption in the clustering by solving:
   \[
   \max_{\{\mathbf{z}_i\}_{i=1}^m} \quad D(C(X_{\text{poisoned}}), C(X)),
   \]
where $D(\cdot, \cdot)$ measures the distance between the clustering dendrograms of the poisoned dataset $X_{\text{poisoned}}$ and the original dataset $X$.

2.Perturbation Strategy: Poisoned points $\{\mathbf{z}_i\}_{i=1}^m$ are crafted to:
   \[
   \mathbf{z}_i = \arg \min_{\mathbf{z}} \left[ \min_{\mathbf{x} \in X} d(\mathbf{z}, \mathbf{x}) \right],
   \]
ensuring they create minimal single-linkage distances and manipulate early cluster merges.

%Explanation
1.Single-Linkage Clustering: In single-linkage clustering, the distance between two clusters is defined as the minimum distance between any pair of points from the two clusters:
     \[
     d(C_i, C_j) = \min_{\mathbf{x} \in C_i, \mathbf{y} \in C_j} d(\mathbf{x}, \mathbf{y}).
     \]
Clusters are iteratively merged based on the smallest linkage distances.

2.Attack Strategy:
- Poisoned points $\{\mathbf{z}_i\}_{i=1}^m$ are designed to influence which clusters merge first by artificially creating smaller linkage distances.
- This disrupts the hierarchy and alters the final clustering results.

3.Optimization:
- For each poisoned point $\mathbf{z}_i$, solve:
     \[
     \mathbf{z}_i = \arg \min_{\mathbf{z}} \left[ \min_{\mathbf{x} \in X} d(\mathbf{z}, \mathbf{x}) \right].
     \]
- This ensures the poisoned points are strategically placed near critical decision boundaries.

4.Output: The poisoned dataset is:
     \[
     X_{\text{poisoned}} = X \cup \{\mathbf{z}_1, \mathbf{z}_2, \ldots, \mathbf{z}_m\}.
     \]

Key Characteristics:
- Target: Hierarchical clustering with single-linkage.
- Effect: Alters the dendrogram and final cluster assignments by exploiting the sensitivity of single-linkage to minimal distances.
- Metric: Perturbations are evaluated by the difference $D(C(X_{\text{poisoned}}), C(X))$, which quantifies the disruption in clustering structure.
