%Input
Let $x$ be the original input image, $y$ be the true label associated with it, and $f_{\theta}$ be the target model. The objective is to create an adversarial example $x^*$ that misleads the model while minimizing perturbations.

%Output
The output of the GenAttack is an adversarial example $x^*$ generated through a generalized optimization process.

%Formula
The GenAttack adversarial attack can be formulated as follows:
1. Initialize the input:
   $x^{(0)} = x$
2. Set parameters for the optimization process:
   - Define the step size $\alpha$ and the maximum perturbation size $\epsilon$.
3. For each iteration $n = 1$ to $N$:
   - Compute the gradient of the loss function:
   $g_n = \nabla_x L(f_{\theta}(x^{(n-1)}), y)$
   where $L$ is the loss function.
   - Update the input with a generalized perturbation:
   $x^{(n)} = x^{(n-1)} - \alpha \cdot g_n$
   - Apply a projection to keep the perturbation within bounds:
   $x^{(n)} = \text{Clip}_{\mathcal{X}}(x^{(n)})$
   ensuring:
   $\|x^{(n)} - x\|_p \leq \epsilon$

4. The final adversarial example is:
   $x^* = x^{(N)}$

%Explanation
The GenAttack adversarial attack is a powerful technique designed to generate adversarial examples by employing a generalized optimization strategy. By iteratively refining the input through the use of gradients, GenAttack focuses on minimizing the loss associated with the target label while keeping the perturbations within a specified limit. The attack's efficiency stems from its ability to adaptively adjust the perturbations based on the model's responses, making it effective against a variety of neural network architectures. The resulting adversarial example $x^*$ can successfully mislead the target model, demonstrating the attack's potency in the realm of adversarial machine learning.
