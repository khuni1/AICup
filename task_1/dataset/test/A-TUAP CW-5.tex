%Input
$X$: The set of data points.\\
$\hat{k}$: The classifier function.\\
$\xi$: The desired $\ell_p$ norm of the perturbation.\\
$\delta$: The desired accuracy on perturbed samples.\\
$v$: The universal perturbation vector.\\
$\text{Err}(Xv)$: The error rate on the perturbed data points $Xv$.\\
$x_i$: A data point in $X$.\\
$\Delta v_i$: The minimal perturbation required to change the classification of $x_i + v$.\\
$\| \cdot \|_2$: The $\ell_2$ norm.\\
$P_{p, \xi}$: The projection operator that projects $v$ onto the $\ell_p$ ball of radius $\xi$.
$y$: Target label.\\
$\gamma$: Constant to avoid over-/underflow in (arc)tanh computations.\\
$c_{\text{init}}$: Initialization of the binary search constant.\\
$c_{\text{upper}}$: Upper bound for the binary search constant.\\
$b_{\text{steps}}$: Number of binary search steps to be performed.\\
$x_{\text{min}}$: Minimum value of input range.\\
$x_{\text{max}}$: Maximum value of input range.\\
$\ell(x)$: Loss function.\\
$\| \cdot \|_2$: $L_2$ norm.

%Output
Universal perturbation vector $v$\\
Perturbation $\hat{r}$\\
Adversarial example $x_{\text{adv}}$


%Formula
Universal Perturbation Algorithm:

1. Initialize the perturbation vector $v \leftarrow 0$.

2. Set the error rate on the perturbed data:
\[
\text{Err}(Xv)
\]

3. While the error rate on the perturbed data is less than or equal to $1 - \delta$:
\[
\text{Err}(Xv) \leq 1 - \delta
\]

4. For each data point $x_i \in X$, do:
\[
\text{If } \hat{k}(x_i + v) = \hat{k}(x_i):
\]
    - Compute the minimal perturbation $\Delta v_i$ that changes the classification of $x_i + v$:
    \[
    \Delta v_i \leftarrow \arg \min_{r} \|r\|_2 \quad \text{s.t.} \quad \hat{k}(x_i + v + r) \neq \hat{k}(x_i)
    \]
    - Update the perturbation:
    \[
    v \leftarrow P_{p, \xi}(v + \Delta v_i)
    \]

5. Repeat the process until the error rate condition is no longer met.

Return the final universal perturbation vector is:
\[
v
\]

Carlini \& Wagner's $L_2$ Attack (Targeted):

1. Normalize the input:
\[
x \leftarrow \frac{x - x_{\text{min}}}{x_{\text{max}} - x_{\text{min}}}
\]
2. Apply the $\text{arctanh}$ transformation:
\[
x \leftarrow \text{arctanh}\left( ((2 \cdot x) - 1) \cdot \gamma \right)
\]
3. Initialize the adversarial example:
\[
x_{\text{adv}} \leftarrow x
\]
4. Set the binary search constants:
\[
c_{\text{lower}} \leftarrow 0, \quad c \leftarrow c_{\text{init}}, \quad c_{\text{double}} \leftarrow \text{True}
\]
5. While there are binary search steps left and $c < c_{\text{upper}}$:
    - Minimize the objective for the current constant:
    \[
    x_{\text{new}} \leftarrow \text{minimize objective}(c)
    \]
    - If the loss is zero and the perturbation is smaller, update the adversarial example:
    \[
    \ell_{\text{min}} \leftarrow \|x_{\text{new}} - x\|_2^2
    \]
    \[
    x_{\text{adv}} \leftarrow x_{\text{new}}
    \]
6. Perform binary search update:
    \[
    \text{If } \ell(x_{\text{new}}) > 0, \text{ set } c \leftarrow c_{\text{lower}}; \quad \text{otherwise set } c \leftarrow c_{\text{upper}}
    \]
7. Update the binary search bounds and decrement the step count:
    \[
    b_{\text{steps}} \leftarrow b_{\text{steps}} - 1
    \]

Return the final adversarial example is:
\[
x_{\text{adv}}
\]

DeepFool for Binary Classifiers:

1. Initialize the input image:
\[
x_0 \leftarrow x
\]
2. While the predicted label for $x_i$ is the same as for the original input:
\[
\text{sign}(f(x_i)) = \text{sign}(f(x_0))
\]
    - Compute the perturbation:
    \[
    r_i \leftarrow - \frac{f(x_i)}{\|\nabla f(x_i)\|_2^2} \nabla f(x_i)
    \]
    - Update the perturbed image:
    \[
    x_{i+1} \leftarrow x_i + r_i
    \]
3. Repeat the process until the classifier’s label changes.

Return the final perturbation is:
\[
\hat{r} = \sum_i r_i
\]


%Explanation
This attack combines a universal perturbation strategy with Carlini Wagner approach for computing minimal perturbations to push data points toward a targeted class, while also maintaining control over the magnitude of the perturbation using the $\ell_p$ norm constraint.
\textbf{Initialization:} Initialize the universal perturbation vector $v$ to 0.\\
\textbf{Iteration Loop:} Continue iterating while the error rate on the perturbed data points $\text{Err}(Xv)$ is less than or equal to $1 - \delta$.\\
\textbf{Perturbation Update for Each Data Point:} For each data point $x_i$ in the set $X$: 
\textbf{If the classifier's prediction for $x_i + v$ is the same as for $x_i$:} 
\textbf{Compute the minimal perturbation $\Delta v_i$ required to change the classification of $x_i + v$:} This involves finding the smallest perturbation $r$ such that the classifier's prediction for $x_i + v + r$ differs from its prediction for $x_i$. 
\textbf{Update the perturbation $v$:} Add $\Delta v_i$ and then project the result onto the $\ell_p$ ball of radius $\xi$ using the projection operator $P_{p, \xi}$.


\textbf{Initialization:} Set the initial perturbed image $x_0$ to the input image $x$. Initialize the iteration counter $i$ to 0.\\
\textbf{Iteration Loop:} Continue iterating while the classifier's sign for $x_i$ is the same as for $x_0$.\\
\textbf{Perturbation Calculation:} Calculate the perturbation $r_i$ as the negative of the classifier's output scaled by the gradient's norm squared. Update the perturbed image $x_{i+1}$ by adding $r_i$ to $x_i$.\\
\textbf{Update:} Increment the iteration counter $i$ by 1.\\
\textbf{Return:} Return the accumulated perturbation $\hat{r}$ as the sum of all $r_i$.


\textbf{Input Normalization:} Normalize the input $x$ to the range [0, 1] and apply the $\text{arctanh}$ transformation to make it suitable for optimization.\\
\textbf{Initialization:} Initialize the adversarial example $x_{\text{adv}}$ and binary search constants.\\
\textbf{Binary Search Loop:} Perform binary search to find the smallest $L_2$ norm perturbation that causes the classifier to misclassify the input as the target label.\\
\textbf{Objective Minimization:} Minimize the objective function for the current binary search constant $c$. Update $x_{\text{adv}}$ if a smaller perturbation is found.\\
\textbf{Binary Search Update:} Adjust the binary search bounds based on the success of the attack.\\
\textbf{Return:} Return the adversarial example $x_{\text{adv}}$ with the smallest perturbation found.
