%Input
Source image $I_v$, target image $I_t$, target label $t$, weight parameters $\alpha$, $\beta$.

%Output
Adversarial loss $L_{\text{Adv}}^t$.

%Formula
The tADV Texture Transfer Attack computes the adversarial loss $L_{\text{Adv}}^t$ as follows:

\[
L_{\text{Adv}}^t = \alpha L_{\text{Att}}(I_v, I_t) + \beta J_{\text{adv}} \left(F(I_v), t \right)
\]

%Explanation
 Texture Transfer Attack tADV, where
$L_{\text{Adv}}^t$ represents the adversarial loss for the texture transfer attack. It combines the texture similarity loss $L_{\text{Att}}(I_v, I_t)$ and the adversarial loss $J_{\text{adv}} \left(F(I_v), t \right)$.

$L_{\text{Att}}(I_v, I_t)$ measures the similarity between the source image $I_v$ and the target image $I_t$ in terms of their texture characteristics. It quantifies how well the texture of $I_v$ is transferred to $I_t$.

$J_{\text{adv}} \left(F(I_v), t \right)$ is the adversarial loss function, measuring the difference between the model's predictions for the adversarially perturbed source image $F(I_v)$ and the target label $t$. It quantifies the discrepancy between the model's output and the desired output.

$F(I_v)$ represents the output of the colorisation model $F$ when applied to the source image $I_v$, generating the colorised version of the image.

$t$ is the target label for the adversarial attack, specifying the desired output of the colorisation model.

$\alpha$ and $\beta$ are weight parameters that control the relative importance of the texture similarity loss and the adversarial loss in the overall adversarial loss $L_{\text{Adv}}^t$. They balance the contributions of the two components.
