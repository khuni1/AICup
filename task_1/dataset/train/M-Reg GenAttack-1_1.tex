%Input
Let $x$ be the original input image, $y$ be the true label associated with it, and $f_{\theta}$ be the target model. The objective is to create an adversarial example $x^*$ that misleads the model while minimizing perturbations.

%Output
The output of the Reg-GenAttack is an adversarial example $x^*$ generated through a generalized optimization process with the introduction of a new constraint: adding a regularization term to the loss function to reduce overfitting.

%Formula
1. Initialize the input:
   $x^{(0)} = x$
2. Set parameters for the optimization process:
   - Define the step size $\alpha$ and the maximum perturbation size $\epsilon$.
3. For each iteration $n = 1$ to $N$:
   - Compute the gradient of the loss function with added L1 regularization:
   $g_n = \nabla_x L(f_{\theta}(x^{(n-1)}), y) + \alpha \cdot |\nabla_x f_{\theta}(x^{(n-1)})|$
   where $L$ is the loss function and $\alpha$ is the regularization parameter.
   - Update the input with a generalized perturbation:
   $x^{(n)} = x^{(n-1)} - \alpha \cdot g_n$
   - Apply a projection to keep the perturbation within bounds:
   $x^{(n)} = \text{Clip}_{\mathcal{X}}(x^{(n)})$
   ensuring:
   $\|x^{(n)} - x\|_p \leq \epsilon$

4. The final adversarial example is:
   $x^* = x^{(N)}$

%Explanation
The (Regularized Generalized (Reg-GenAttack) Attack variant with L1 regularization improves upon the original attack by introducing a constraint that reduces overfitting. This new approach maintains the core principle of minimizing perturbations while misleading the model, but adds a layer of regularization to prevent overfitting. The result is an adversarial example $x^*$ that is not only effective in misleading the target model but also generalizes well across multiple instances.