%Input
Original input sequence (e.g., text or image data).
Target model (e.g., classifier or regressor).
True label of the input.
Perturbation limit for adversarial perturbation.
Loss function to minimize for successful attack.
Maximum number of iterations.

%Output
The Kantchelian Gradient-Based Attack is a variant of the original perturbation core. It generates adversarial examples by iteratively perturbing the input data using gradients of the loss function with respect to the input.

%Formula
1. Initialization:
   \[
   \mathbf{x}_{\text{adv}}^{(0)} = \mathbf{x}_{\text{original}}.
   \]

2. Perturbation Update:
   Compute the perturbation as the solution to the optimization problem:
   \[
   \mathbf{x}_{\text{adv}}^{(t+1)} = \mathbf{x}_{\text{adv}}^{(t)} + \alpha \cdot \nabla_{\mathbf{x}} \mathcal{L}(f, \mathbf{x}_{\text{adv}}^{(t)}, y_{\text{true}}),
   \]
   where $\alpha$ is the step size for each update.

3. Perturbation Constraint:
   Ensure the perturbation remains within the limit $\epsilon$:
   \[
   \mathbf{x}_{\text{adv}}^{(t+1)} = \text{clip}(\mathbf{x}_{\text{adv}}^{(t+1)}, \mathbf{x}_{\text{original}} - \epsilon, \mathbf{x}_{\text{original}} + \epsilon).
   \]

4. Stopping Condition:
   Terminate the attack when:
   \[
   f(\mathbf{x}_{\text{adv}}^{(t+1)}) \neq y_{\text{true}}, \quad \text{or when} \, t \geq N.
   \]
Explanation: The Kantchelian Adversarial Attack is a variant of the original perturbation core. It generates adversarial examples by iteratively perturbing the input data using gradients of the loss function with respect to the input.

%Explanation
This attack is logically derived from the perturbation core described in the input provided. The main difference between this variant and the original perturbation core is that it uses gradient-based optimization to craft adversarial perturbations, which allows for more efficient and targeted attacks. Additionally, the Kantchelian Attack introduces a new constraint on the perturbation limit, ensuring that the generated adversarial examples are within a specific range of the original input data.