%Input
Let $x$ be the original input text, $y$ be the true label, and $f_{\theta}$ be the target model. The TextBugger attack generates adversarial examples by introducing minimal changes to the input text.
Input: Identify a subset of words in the input text with high semantic importance and introduce character-level perturbations only to these words.

%Output
The output of the Semantically Targeted TextBugger (ST-TB) variant is a modified text input $x^*$ that aims to mislead the model while preserving the original meaning of the text. This variant focuses on strategically targeting critical words in the input, potentially making it more difficult for the model to distinguish between legitimate and adversarial examples.

%Formula
1. Initialize the input text and true label:
   $
   (x, y).
   $
2. Identify a subset of words with high semantic importance using a measure such as word embedding similarity or part-of-speech tagging:
   $
   S = \{w_1, w_2, \ldots, w_n\}.
   $
3. For each word $w_i$ in the input text:
   - Generate a set of candidate modifications $S(w_i)$ based on character-level perturbations:
   $
   S(w_i) = \{w_{i1}, w_{i2}, \ldots, w_{im}\}.
   $
4. For each candidate substitution $w_{ij}$:
   - Construct a modified input:
   $
   x' = (w_1, \ldots, w_{i-1}, w_{ij}, w_{i+1}, \ldots, w_n).
   $
   - Evaluate the model's prediction:
   $
   \hat{y} = f_{\theta}(x').
   $
   - If $\hat{y} \neq y$, then $x'$ is a candidate adversarial example.
5. The goal is to find:
   $
   x^* = \arg\max_{x'} \text{Prob}(f_{\theta}(x') \neq y),
   $
   ensuring minimal modification to the original text while targeting critical words.

%Explanation
The Semantically Targeted TextBugger (ST-TB) variant modifies the TextBugger attack by focusing on a specific subset of words with high semantic importance. This approach aims to increase the attack's effectiveness by strategically targeting these critical words, potentially making it more challenging for the model to distinguish between legitimate and adversarial examples.