%Input
$x$: The original input image or data point.
$y$: The true class label of the input $x$.
$F(x)$: The output of the classifier for input $x$ (typically the logits or class probabilities).
$\nabla F(x)$: The gradient of the classifier's output with respect to $x$.
$\nabla^2 F(x)$: The Hessian (second derivative) of the classifier's output with respect to $x$.
$x_{\text{adv}}$: The adversarial sample, initialized to $x$.
$\eta$: The step size for the perturbation.
$i_{\text{max}}$: The maximum number of iterations.
$\text{clip}(x_{\text{adv}})$: A function to clip $x_{\text{adv}}$ to the valid image space, ensuring pixel values are within a specified range.

%Output
The final adversarial sample $x_{\text{adv}}$, which is designed to mislead the classifier while maintaining the original structure of $x$.

%Formula
Initialize the adversarial sample: 
\( x_{\text{adv}} \leftarrow x \).

For each iteration \( i = 1, \ldots, i_{\text{max}} \):
\[
\nabla F(x_{\text{adv}}), \quad \nabla^2 F(x_{\text{adv}})
\]
Solve Newton's method update step:
\[
\delta = - (\nabla^2 F(x_{\text{adv}}))^{-1} \nabla F(x_{\text{adv}})
\]
Update the adversarial sample:
\[
x_{\text{adv}} \leftarrow x_{\text{adv}} + \eta \cdot \delta
\]
Clip the adversarial sample to the valid image space:
\[
x_{\text{adv}} \leftarrow \text{clip}(x_{\text{adv}})
\]
Increment the iteration counter:
\[
i \leftarrow i + 1
\]

\subsection*{Return}
The final adversarial sample is:
\[
x_{\text{adv}}
\]

%Explanation
The Newton-Fool adversarial attack uses second-order optimization (Newton’s method) to find adversarial perturbations. In each iteration:
1. The gradient \( \nabla F(x_{\text{adv}}) \) and Hessian \( \nabla^2 F(x_{\text{adv}}) \) of the classifier are computed at the current adversarial sample.
2. The perturbation \( \delta \) is computed using the Newton’s method formula, where the gradient is scaled by the inverse of the Hessian.
3. The adversarial sample is updated by adding the perturbation scaled by the step size \( \eta \).
4. The updated sample is clipped to ensure it remains a valid image (i.e., all pixels are within the valid range).
5. The process repeats until the maximum number of iterations is reached, producing the final adversarial example \( x_{\text{adv}} \).