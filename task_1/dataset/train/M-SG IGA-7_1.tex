%Input
Let \( f \) be the target model, \( x \) be the original input data point, \( y \) be the true label, and \( \theta \) be the model's parameters.  
Define the maximum allowable perturbation \( \epsilon \), the standard deviation \( \sigma \) of the Gaussian noise, and the number of iterations \( N \).  
 

%Output
The output of the M-Gaussian Noise-Based Iterative Perturbation Attack is an adversarial example $x^*$ that misclassifies the input while incorporating Gaussian noise. The new attack maintains the core principle of the original attack but introduces a scoring function to guide the perturbation process.

%Formula

1. Initialize the input:
   $
   x^{(0)} = x.
   $
2. Set parameters for the noise generation:
   - Define the maximum perturbation size $\epsilon$, the standard deviation $\sigma$ of the Gaussian noise, and the number of iterations $N$.
3. For each iteration $n = 1$ to $N$:
   - Compute the model's prediction:
   $
   \hat{y}^{(n)} = f_{\theta}(x^{(n-1)}).
   $
   - Generate Gaussian noise:
   $
   \text{noise} \sim \mathcal{N}(0, \sigma^2 I),
   $
   where $I$ is the identity matrix.
   - Update the input by adding the noise with a score-based update function:
   $
   x^{(n)} = x^{(n-1)} + \text{noise} \cdot f(x^{(n-1)}, y, \theta),
   $
   where $f$ is a scoring function that evaluates the model's confidence in its prediction.
   - Apply clipping to ensure the perturbation stays within bounds:
   $
   x^{(n)} = \text{Clip}_{\mathcal{X}}(x^{(n)}),
   $
   ensuring:
   $
   \|x^{(n)} - x\|_p \leq \epsilon.
   $

4. The final adversarial example is:
   $
   x^* = x^{(N)}.
   $

%Explanation
The Score-Guided Iterative Gaussian Perturbation Attack (SG-IGA) generates adversarial examples by adding Gaussian noise to the original image in an iterative manner, guided by a scoring function that enhances the model's confidence-based approach. This variant improves upon the original attack by incorporating a score-based update function that encourages more effective perturbations while maintaining control over the perturbation size and bounds.