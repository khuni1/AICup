%input
The input is the adversarial perturbation $\delta$ and the feature probabilities $\overline{l}_i(\delta)$ for each class.

%output
The output is the loss value computed for the adversarial perturbation $\delta$.

%formula
The loss function is given by:
$
\text{Loss} = -\log \left( \prod_{i=1}^{K} \overline{l}_i(\delta) \right)
\text{ such that } \| \delta \|_8 < \xi$

where:
\begin{equation*}
\text{Loss} = -\log \left( \prod_{i=1}^{K} \overline{l}_i(\delta) \right)
\end{equation*}

is the negative log-likelihood of the modified feature probabilities, and $\| \delta \|_8 < \xi$

is the constraint on the $L_8$ norm of the perturbation $\delta$.

%Explanation
Logit Space where the loss function aims to maximize the likelihood of incorrect classifications by minimizing the negative log-likelihood of the modified feature probabilities. The constraint $\| \delta \|_8 < \xi$ ensures that the perturbation $\delta$ remains within a bounded magnitude to keep the adversarial attack imperceptible. The parameter $\xi$ represents the maximum allowable perturbation magnitude, controlling the strength of the attack.
