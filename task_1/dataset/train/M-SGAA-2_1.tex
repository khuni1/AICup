%Input
The input is the feature \( x_i \) of the input sample \( x \), the target class \( t \), and the output class \( o \). Additionally, the current pixel value and the gradient of the probability of the target class with respect to the pixel values are used.

%Output
The adversarial example $ x^* $ is generated by selectively perturbing the input sample $ x $ based on a sensitivity score function $ s(x_i) $. The resulting adversarial perturbation $ \delta^* $ is constrained to modify only the most influential pixels, ensuring a more targeted attack.

\[
x^* = x + \delta^*, \quad \text{such that } f(x^*) = y_t, \quad \|\delta^*\| \text{ is minimized and constrained by } s(x_i) > 0
\]

%Formula
The updated formula for the perturbation $\delta$ can be modified by incorporating an additional constraint, ensuring that the perturbation only affects pixels with a certain sensitivity score.

\begin{align*}
\text{New Pixel Value} = \text{Current Pixel Value} + \epsilon \cdot \nabla_{\text{pixel}} p(y_t | x) \\
s(x_i) &= \frac{\partial y_t}{\partial x_i} - s_o\\
\delta^{(n)} = \text{Clip}_{\mathcal{X}} \left( \delta^{(n-1)} + \alpha \cdot \text{sign} \left( s(y_t | x) \right) \right)
\end{align*}

%Explanation
This variant Sensitivity-Guided Adversarial Attack (SGAA) is different from the main perturbation core by incorporating a sensitivity score $s(x_i)$ that quantifies the importance of each feature in influencing the network's decision. The new attack only updates pixels with non-zero sensitivity scores, making it more targeted and stealthy. Additionally, the updated formula for the perturbation $\delta$ incorporates an additional constraint to ensure that the perturbation only affects pixels with high sensitivity scores. This improves the overall effectiveness of the attack while maintaining its core principle.