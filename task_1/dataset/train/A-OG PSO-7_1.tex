%Input
$x$: Original image \\
$r$: Perturbation applied to $x$ \\
$f(x + r)$: Objective function to minimize \\
$O$: Oracle that outputs a candidate list \\
$\text{score}_{\text{top}}$: Confidence score of the top-ranked item in the candidate list \\
$\text{score}_{\text{target}}$: Confidence score of the target in the candidate list \\
$\text{rank}$: Rank of the target in the candidate list \\
$\text{maxObjective}$: A large value when the target is not in the top three candidates. \\
$\text{image}_{\text{original}}$: Original image \\
$candidates$: Candidate list returned by oracle $O$ \\
$target$: Target to be impersonated \\
$numParticles$: Number of particles in the PSO swarm \\
$epochmax$: Maximum number of PSO iterations \\
$seed$: Set of particles providing an initial solution.

%Output
Movement of the particle swarm to minimize the objective function based on candidates provided by the oracle, or optimal perturbation for impersonation.


%Formula
$f(x) = 
\begin{cases}
\min(f(x + r)) & \text{if } target \in candidates \\
\max(f(x + r)) + \text{maxObjective} & \text{otherwise}
\end{cases}$

%Explanation
This variant Oracle-Guided PSO Attack (OG-PSO) is different from the main perturbation core in that it incorporates an additional constraint. The new objective function $f(x)$ now favors the minimum value when the target appears among the candidates, while applying a maximum penalty when the target is not present. This modification introduces a more nuanced scoring system, encouraging particles to converge on successful impersonations while discouraging non-optimal solutions.