%Input
The goal is to generate adversarial examples using an iterative process, where each iteration involves updating the adversarial example based on a gradient update rule that incorporates momentum. The gradient is scaled by the inverse of the Hessian matrix of the loss function to adjust the update direction and magnitude.

%Formula
$g_{m+1} = \mu \cdot g_m + \nabla L(x_m, y_c; \theta) \cdot \nabla(L(x_m, y_c; \theta))^{-1} \label{eq:gmplus1}$

Update Rule: $g_{m+1}$ This represents the updated gradient at the \((m+1)\)-th iteration of the adversarial example generation process.

Momentum Term:$\mu \cdot g_m$ This term represents the momentum contribution to the gradient update, where \(g_m\) is the gradient at the \(m\)-th iteration and \(\mu\) is the momentum coefficient. It allows the gradient to accumulate over iterations, enabling smoother and more stable updates.

Gradient of Loss Function:$\nabla L(x_m, y_c; \theta)$ This is the gradient of the loss function \(L\) with respect to the input \(x_m\), evaluated at the current adversarial example \(x_m\) and its corresponding true class label \(y_c\), parameterized by \(\theta\).

Inverse Gradient Matrix: $\nabla(L(x_m, y_c; \theta))^{-1}$ This term represents the inverse of the Hessian matrix of the loss function \(L\) with respect to the input \(x_m\), evaluated at the current adversarial example \(x_m\) and its corresponding true class label \(y_c\), parameterized by \(\theta\). 
It is used to scale the gradient update.$x_{m+1} = \text{clip} \left( x_m + \alpha \cdot \text{sign}(g_{m+1}) \right) \label{eq:xmplus1}$

Adversarial Example Update: $x_{m+1}$ This represents the adversarial example at the \((m+1)\)-th iteration of the adversarial example generation process.
Current Adversarial Example $x_m$ This is the current adversarial example obtained at the \(m\)-th iteration.
Step Size: $\alpha$ This is the step size or perturbation magnitude, which controls the size of the update applied to the adversarial example.

Gradient Update:$g_{m+1}$ This is the updated gradient at the \((m+1)\)-th iteration, obtained using the update rule described in Equation \ref{eq:gmplus1}.

Sign Function: ${sign}(g_{m+1})$ This function returns the sign of each element of the gradient \(g_{m+1}\), indicating the direction of the update.

Clip Function: ${clip}(\cdot)$ This function clips the values of its argument to a specified range. In this context, it is used to ensure that the perturbed adversarial example remains within a permissible range of values, typically the valid range of input features.


%Output
The adversarial example \(x_{m+1}\) after the \((m+1)\)-th iteration, which is generated by applying the gradient update and clipping the resulting values to ensure they stay within a valid range. The gradient update includes a momentum term to stabilize and smooth the adversarial example generation process.

%Explanation
Momentum-Based Inverse-Matrix process involves several key components:
The gradient \(g_{m+1}\) is updated using the momentum term \(\mu \cdot g_m\) and the gradient of the loss function \(\nabla L(x_m, y_c; \theta)\), scaled by the inverse of the Hessian matrix.
The adversarial example \(x_{m+1}\) is updated by adding the signed gradient \(g_{m+1}\) scaled by the step size \(\alpha\).
he `clip` function ensures that the adversarial example \(x_{m+1}\) remains within a permissible range of values, preventing the perturbation from exceeding predefined bounds.
