%Input
\[
\text{Given problem: Minimize } f(x_1, x_2) = x_1 + x_2 \text{ subject to } x_1^2 + x_2^2 \leq 2
\]

%Output
Optimal solution: 
\[
(x_1, x_2) = (0.9, 0.1)
\]
Objective value: 
\[
f(0.9, 0.1) = 1.0
\]

%Formula:
\[
\mathcal{L}(x_1, x_2, \lambda, \mu_1, \mu_2) = x_1 + x_2 + \lambda (x_1^2 + x_2^2 - 2) - \mu_1 x_1 - \mu_2 x_2
\]

%Explanation
This Constrained Exploration Perturbation (CEP) modified perturbation core is an adaptation of the original Lagrange multiplier method. The new constraints are given by $0.9 \leq x_1, 0.1 \leq x_2$, representing a reduced solution space while maintaining feasibility and non-negativity conditions for both variables. By reducing the constraint values, we introduce some infeasibility into our initial guesses to increase exploration of more optimal regions of the objective function $f(x_1,x_2) = x_1 + x_2$.

The variant differs from the main perturbation core by introducing reduced constraint values, allowing for slightly infeasible solutions and potentially increasing the effectiveness of the attack.