%Input
The input consists of the initial input $x_m$, the gradient of the loss function $\nabla L(x_m, y_c; \theta)$, the step size $\alpha$, and the momentum term $\mu$.

%Output
The output is the updated input $x_{m+1}$ after one iteration of the algorithm, and the gradient $g_{m+1}$ used to compute the update.

%Formula
The iterative update rules for the gradient and the input are given by:

\begin{align}
g_{m+1} &= \mu \cdot g_m + \frac{\nabla L(x_m, y_c; \theta)}{\|\nabla L(x_m, y_c; \theta)\|_1} \\
x_{m+1} &= \text{clip} \left( x_m + \alpha \cdot \text{sign}(g_{m+1}) \right)
\end{align}

where:
\begin{equation*}
\text{clip}(\cdot)
\end{equation*}
is a function that constrains the value of its argument to stay within a specified range.

\begin{equation*}
\mu
\end{equation*}
is a momentum term that helps in smoothing the gradient updates and accelerating convergence.

\begin{equation*}
\nabla L(x_m, y_c; \theta)
\end{equation*}
is the gradient of the loss function $L$ with respect to the input $x_m$ and the true class $y_c$, parameterized by $\theta$.

\begin{equation*}
\|\nabla L(x_m, y_c; \theta)\|_1
\end{equation*}
is the $\ell_1$-norm of the gradient, which normalizes the gradient to ensure it does not become excessively large.

\begin{equation*}
\text{sign}(g_{m+1})
\end{equation*}
is the sign function applied to the gradient $g_{m+1}$, which extracts the direction of the gradient.

\begin{equation*}
\alpha
\end{equation*}
is the step size for each iteration.

%Explanation
Momentum Iterative Method (MIM) where the iterative method involves updating the gradient $g_{m+1}$ by incorporating a momentum term $\mu$ that smooths the gradient updates. The gradient is normalized using the $\ell_1$-norm to prevent it from becoming excessively large. The input $x_{m+1}$ is updated by adding a perturbation in the direction of the sign of the gradient $g_{m+1}$, scaled by the step size $\alpha$. The $\text{clip}$ function ensures that the updated input stays within a specified range, maintaining the constraints of the adversarial perturbation.
