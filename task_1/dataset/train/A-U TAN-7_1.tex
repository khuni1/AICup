%Input
%Input
$X$: Input dataset containing $n$ samples.
$t$: Target class for the adversarial perturbation.
$v$: Universal perturbation vector.
$\Delta v_i$: Perturbation for each data point $x_i$.
$P_{p, \xi}$: Projection operator onto the $\ell_p$ ball with radius $\xi$.
$\hat{k}$: Classifier's prediction function.
$\delta$: Confidence margin for adversarial misclassification.


%Output
The output of the TUAP is a universal perturbation vector $v$ added to multiple data points $x_i$, resulting in adversarial examples that can deceive the model.

%Formula
1. Initialize the perturbation vector $v \leftarrow 0$.
2. While the error rate on the perturbed data is less than or equal to $1 - \delta$:
\[
\text{Err}(Xv) \leq 1 - \delta
\]
3. For each data point $x_i \in X$, do:
    - If the classifier's prediction for $x_i + v$ is not the target class $t$:
        \[
        \hat{k}(x_i + v) \neq t
        \]
            - Compute the minimal perturbation $\Delta v_i$ that sends $x_i + v$ to the target class $t$:
            \[
            \Delta v_i \leftarrow \arg \min_{r} \|r\|_2 \quad \text{s.t.} \quad \hat{k}(x_i + v + r) = t
            \]
            - Update the perturbation:
            \[
            v \leftarrow P_{p, \xi}(v + \Delta v_i)
            \]

4. Repeat the process until the error rate condition is no longer met.

\subsection*{Return}
The final targeted universal perturbation vector is:
\[
v
\]

%Explanation
Universal $\ell_p$ Norm Constrained Targeted Adversarial Perturbation (U-TAN) is a variant of the original attack that incorporates $\ell_p$ norm projection to enhance its effectiveness. The key difference between this variant and the main perturbation core lies in the use of projection operators to constrain the perturbations within a specific norm limit, which improves the robustness and universality of the attack. This modification allows for more effective and targeted adversarial examples that can generalize across multiple data points.