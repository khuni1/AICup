%Input
$H^{(1)}_{ar}, H^{(2)}_{ar}, \ldots, H^{(N)}_{ar}$: $N$ realizations of channels. \\
$\mathbf{r}_{tr}$: Input representation. \\
Model of the classifier: The classifier model for adversarial attack.

%Output
$\Delta = \mathbf{U} \Sigma \mathbf{V}^T$: Singular Value Decomposition of $\Delta$. \\
$\mathbf{v}_1 = \mathbf{V}\mathbf{e}_1$: First principal component. \\
$\delta_{\text{limited}} = \sqrt{P_{\text{max}}} \mathbf{v}_1$: Limited perturbation in the first principal direction.

%Formula
\subsection*{Principal Component Analysis (PCA)}
Initialize the perturbation matrix:
\[
\Delta \leftarrow 0
\]

For each realization $n = 1, \ldots, N$:
\[
\delta^{(n)} = \text{White-box attack result using } \mathbf{r}_{tr} \text{ and } H^{(n)}_{ar}
\]
Stack the perturbation:
\[
\Delta \leftarrow \Delta + \delta^{(n)}
\]

Perform Singular Value Decomposition (SVD) on $\Delta$:
\[
\Delta = \mathbf{U} \Sigma \mathbf{V}^T
\]
Extract the first principal component:
\[
\mathbf{v}_1 = \mathbf{V} \mathbf{e}_1
\]

Calculate the final perturbation:
\[
\delta_{\text{limited}} = \sqrt{P_{\text{max}}} \mathbf{v}_1
\]

\subsection*{Return}
The final adversarial perturbation is:
\[
\delta_{\text{limited}}
\]


%Explanation
Principal Component Analysis (PCA) and is designed to operate under limited channel information, the attack leverages limited channel information by computing the principal direction of adversarial perturbations through PCA.
