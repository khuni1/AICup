%Input:
Original input $x$, ground-truth label $y$, total allowed perturbation $\epsilon$, number of iterations $T$, decay factor $\mu$.

%Output:
Adversarial example $x^*_T$.

%Formula:
The MI-FGSM algorithm generates an adversarial example $x^*$ by iteratively updating the original input $x$ as follows:

$
x^*_T = x + \sum_{t=0}^{T-1} \alpha \cdot \text{sign} \left( \sum_{i=0}^{t} \frac{\mu^i \nabla_x J(x^*_{t-i}, y)}{\| \nabla_x J(x^*_{t-i}, y) \|_1} \right)
$

%Explanation:
Momentum Iterative (MIFGM) defined as $\alpha = \frac{\epsilon}{T}$ is the step size.

$\nabla_x J(x^*_t, y)$ is the gradient of the loss function $J$ with respect to $x$ at iteration $t$, evaluated at the current adversarial example $x^*_t$ and the ground-truth label $y$.

$\| \nabla_x J(x^*_t, y) \|_1$ is the $L_1$ norm of the gradient, used for normalization.

The inner sum $\sum_{i=0}^{t} \frac{\mu^i \nabla_x J(x^*_{t-i}, y)}{\| \nabla_x J(x^*_{t-i}, y) \|_1}$ accumulates the gradients with exponential decay, effectively incorporating momentum into the gradient-based attack.
