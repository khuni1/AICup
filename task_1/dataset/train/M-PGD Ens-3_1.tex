%Input 
Let $x$ be the original input sample, and let $f_{\theta}$ represent the target model parameterized by \(\theta\). The goal is to generate an adversarial perturbation \(\delta\) that maximizes the loss while considering an ensemble of loss functions.  

The optimization problem is formulated as:  
\[
\underset{\delta}{\text{maximize}} \quad L(f_{\theta}(x + \delta), y) + \beta \cdot g(F_1, F_2, \dots, F_n)
\]
subject to the constraint:  
\[
\|\delta\|_p \leq \epsilon
\]
where \(L\) represents the primary loss function, \(g\) is an ensemble-based loss function, \(\beta\) is a trade-off parameter, and \(\epsilon\) defines the perturbation budget.


%Output
The PGD Ensemble Attack variant is a logically derived extension of the original perturbation core. It combines the principles of ensemble-based optimization with the Fast-Universal attack, aiming to improve the robustness and diversity of generated adversarial examples.

%Formula
$\delta^{(0)} = 0$
for $n = 1 \text{ to } N: \quad\n\delta^{(n)} = \text{Clip}_{\mathcal{X}} \left( \delta^{(n-1)} + \alpha \cdot \text{sign} \left( \nabla_\delta L(f_\theta(x + \delta^{(n-1)}), y) \right) + \beta \cdot g^n(F_1, F_2, \dots, F_n) \right)$

where $g^n(F_1, F_2, \dots, F_n)$ represents an ensemble of loss functions evaluated at the current iteration $n$, and $\beta$ is a hyperparameter controlling the trade-off between the perturbation and the ensemble-based optimization.

%Explanation
The Projected Gradient Descent Ensemble Attack (PGD Ens) variant extends the original Fast-Universal attack by incorporating an ensemble-based optimization component. The addition of the ensemble loss function, $g^n(F_1, F_2, \dots, F_n)$, allows the attack to adaptively adjust the perturbation $\delta^{(n)}$ based on the performance of individual components within the ensemble. This modification enables the generated adversarial examples to be more robust and diverse across different parts of the input space.

The PGD Ensemble Attack variant differs from the original PGD Fast-Universal attack by incorporating an ensemble-based optimization component, which adaptsively adjusts the perturbation based on the performance of individual components within the ensemble. This modification improves the robustness and diversity of generated adversarial examples.