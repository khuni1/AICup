%Input
The input is the original sample \(x\) and the initial perturbation \(\delta\). The perturbation is then subjected to randomization and chaotic transformations.

%Output
The output is the updated perturbation \(\delta_{\text{final}}\) applied to the input \(x\), resulting in the adversarial example \(x_{\text{new}} = x + \delta_{\text{final}}\).

%Formula
The perturbation update follows these steps:

Randomization Spell Generate a random perturbation \(\delta_{\text{random}}\) from a Gaussian distribution:

\[
\delta_{\text{random}} \sim \mathcal{N}(0, \sigma)
\]

Hexing with Mysterious Filters Apply a chaotic transformation \(f(\cdot)\) (e.g., logistic map) to the random perturbation:

\[
\delta_{\text{hexed}} = f(\delta_{\text{random}}), \quad f(x) = 4x(1 - x)
\]

Witch’s Cauldron Blending Combine multiple perturbations with random weights:

\[
\delta_{\text{final}} = \sum_{i} \omega_i \cdot \delta_i
\]

Unpredictable Update Rule Apply the final perturbation to the input:

\[
x_{\text{new}} = x_{\text{old}} + \delta_{\text{final}}
\]

Hexed Decision Choose the perturbation that maximizes the classification uncertainty:

\[
\text{Chaos Metric} = \sum_j \left| \frac{\partial y_j}{\partial x_i} \right|
\]

%Explanation
WitchCraft or Magical where the input data \(x\) undergoes a sequence of chaotic transformations to generate an adversarial perturbation. The process begins by creating a random perturbation \(\delta_{\text{random}}\) from a Gaussian distribution, followed by applying a chaotic map (e.g., the logistic map) to introduce non-linearities and unpredictability into the perturbation. Multiple perturbations are blended using random weights to form a final perturbation \(\delta_{\text{final}}\).

The perturbation is then applied to the input, resulting in a new adversarial example \(x_{\text{new}} = x + \delta_{\text{final}}\). The final perturbation is chosen to maximize the **Chaos Metric**, which measures the uncertainty of the model’s classification output. This ensures that the adversarial example causes unpredictable and erratic behavior in the model, resembling the chaotic and uncontrollable nature of a "witchcraft" approach in adversarial machine learning.
