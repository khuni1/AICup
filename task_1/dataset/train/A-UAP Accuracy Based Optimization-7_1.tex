%Input
$X$: The set of data points. 
$\hat{k}$: The classifier function. 
$\xi$: The desired $\ell_p$ norm of the perturbation. 
$\delta$: The desired accuracy on perturbed samples. 
$v$: The universal perturbation vector. 
$\text{Err}(Xv)$: The error rate on the perturbed data points $Xv$. 
$x_i$: A data point in $X$. 
$\Delta v_i$: The minimal perturbation required to change the classification of $x_i + v$. 
$\| \cdot \|_2$: The $\ell_2$ norm. 
$P_{p, \xi}$: The projection operator that projects $v$ onto the $\ell_p$ ball of radius $\xi$.

%Output 
Universal perturbation vector $v$

%Formula
1. Initialize the perturbation vector $v \leftarrow 0$.
2. While the error rate on the perturbed data is less than or equal to $1 - \delta$:
\[
\text{Err}(Xv) \leq 1 - \delta
\]
3. For each data point $x_i \in X$, do:
    - If the classifier's prediction for $x_i + v$ equals the prediction for $x_i$:
    \[
    \hat{k}(x_i + v) = \hat{k}(x_i)
    \]
        - Compute the minimal perturbation $\Delta v_i$ that sends $x_i + v$ to the decision boundary:
        \[
        \Delta v_i \leftarrow \arg \min_{r} \|r\|_2 \quad \text{s.t.} \quad \hat{k}(x_i + v + r) \neq \hat{k}(x_i)
        \]
        - Update the perturbation:
        \[
        v \leftarrow P_{p, \xi}(v + \Delta v_i)
        \]
4. Repeat the process until the error rate condition is no longer met.

Return the final universal perturbation vector is:
\[
v
\]

%Explanation: 
This variant of Universal Adversarial Perturbation (UAP) modifies the original method by incorporating a new accuracy-based optimization loop that iteratively updates the perturbation vector $v$ to achieve a desired accuracy $\delta$. The key difference between this variant and the main UAP method is the addition of an accuracy-based stopping criterion, which allows the algorithm to focus on generating perturbations that result in a specific error rate. This modification can lead to more targeted attacks that are tailored to the specific requirements of the problem at hand.

