%Input
$\mathbf{r}_{tr}$: Input representation. \\
$E$: Total number of iterations. \\
$P_{\text{max}}$: Power constraint limit for the attack. \\
$y_{\text{true}}$: Ground truth class label for the input. \\
Model of the classifier: The target model.

%Output: 
$\delta_{\text{meta}} = \frac{\nabla_{\mathbf{x}} L(\theta, \mathbf{x}, y_{\text{true}})}{\|\nabla_{\mathbf{x}} L(\theta, \mathbf{x}, y_{\text{true}})\|_2}$
$\mathbf{x} \leftarrow \mathbf{x} + \sqrt{\frac{P_{\text{max}}}{E}} H_{ar} \delta_{\text{meta}}$
$\Delta \leftarrow \Delta + \sqrt{\frac{P_{\text{max}}}{E}} \delta_{\text{meta}}$
$\delta_{\text{meta}} = \sqrt{P_{\text{max}}} \frac{\Delta}{\|\Delta\|_2}$

%Formula: 
The Meta-Learning Based Perturbation Variant is formulated as follows:

1. Initialize the sum of gradients and input:
   $ \Delta \leftarrow 0 $, $ \mathbf{x} \leftarrow \mathbf{r}_{tr} $

2. For each iteration $ e = 1, \ldots, E $ (where $ E $ is the total number of iterations):
   - Compute the normalized gradient:
     $ \delta_{\text{meta}} = \frac{\nabla_{\mathbf{x}} L(\theta, \mathbf{x}, y_{\text{true}})}{\|\nabla_{\mathbf{x}} L(\theta, \mathbf{x}, y_{\text{true}})\|_2} $
   - Update the input:
     $ \mathbf{x} \leftarrow \mathbf{x} + \sqrt{\frac{P_{\text{max}}}{E}} H_{ar} \delta_{\text{meta}} $
   - Accumulate the gradient:
     $ \Delta \leftarrow \Delta + \sqrt{\frac{P_{\text{max}}}{E}} \delta_{\text{meta}} $

3. After all iterations, compute the meta perturbation:
   $ \delta_{\text{meta}} = \sqrt{P_{\text{max}}} \frac{\Delta}{\|\Delta\|_2} $

%Explanation
The algorithm iterates over $E$ steps, updating the input $\mathbf{x}$ at each iteration based on the normalized gradient. However, this variant introduces a meta-learning component by storing and reusing the accumulated gradients $\Delta$. This allows the attack to adapt to the model's behavior and improve its effectiveness across multiple iterations.

Summary: The Meta-Learning Based Perturbation Variant is distinct from the original Naive Non-Targeted Attack in that it incorporates meta-learning capabilities, which enable the attack to learn from its previous attacks and improve its performance over time. This variant introduces a new layer of adaptability to the perturbation strategy, making it more effective against evolving models.