%Input:
Adversarial loss function $\mathcal{L}_{\text{adv}}(x, f)$, flow regularization loss $\mathcal{L}_{\text{flow}}(f)$, trade-off parameter $\tau$.

%Output:
Optimal function $f^*$.

%Formula
The Spatial Transformation attack computes the optimal function $f^*$ as follows:

\[
f^* = \text{argmin}_f \left( \mathcal{L}_{\text{adv}}(x, f) + \tau \mathcal{L}_{\text{flow}}(f) \right)
\]

%Explanation
Spatial Transformation Attack StAdv where $f^*$ represents the optimal function that minimizes the combined loss function.

$f$ denotes the space of functions over which the optimization is performed. The goal is to find the function $f^*$ that minimizes the combined loss.

$\mathcal{L}_{\text{adv}}(x, f)$ is the adversarial loss term, measuring the discrepancy between the predictions of the model $f$ and the ground truth or target label $x$. It quantifies how well the model resists adversarial perturbations or generates adversarial examples.

$\mathcal{L}_{\text{flow}}(f)$ is the flow regularization loss, applied to the model $f$ to encourage smoothness or consistency in the predictions. It helps prevent overfitting and improve generalization.

$\tau$ is the trade-off parameter that controls the balance between the adversarial loss and the flow regularization loss. It determines the relative importance of each component in the overall optimization objective.

$\text{argmin}_f$ finds the function $f^*$ that minimizes the combined loss function over the function space $f$.
