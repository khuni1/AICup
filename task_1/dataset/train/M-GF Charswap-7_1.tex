%Input
Let \( x \) be the original input text, \( y \) be the true label, and \( f_{\theta} \) be the target model. The PruthiKeyboard Charswap attack generates adversarial examples by swapping characters in the input text to deceive the model.

%Output
The output of the Charswap attack is a modified text input \( x^* \) that aims to mislead the model while maintaining a degree of semantic similarity to the original text.

%Formula
1. Initialize the input text and true label:
   $
   (x, y).
   $
2. Tokenize the input text:
   $
   x = (w_1, w_2, \ldots, w_n).
   $
3. For each character \( c_i \) in the input text:
   - Identify potential characters to swap based on keyboard layout:
   $
   S(c_i) = \{c_{i1}, c_{i2}, \ldots, c_{im}\}.
   $
4. For each candidate character \( c_{ij} \):
   - Construct a modified input by swapping characters:
   $
   x' = (w_1, \ldots, w_{i-1}, c_{ij}, w_{i+1}, \ldots, w_n).
   $
   - Evaluate the model's prediction:
   $
   \hat{y} = f_{\theta}(x').
   $
   - If \( \hat{y} \neq y \), then \( x' \) is a candidate adversarial example.
5. The goal is to find:
   $
   x^* = \arg\max_{x'} \text{Prob}(f_{\theta}(x') \neq y),
   $
   ensuring that the number of swaps is minimized.

%Explanation
The Gradient-Free Charswap Attack (GF-Charswap) variant generates adversarial examples by strategically swapping characters in the input text \( x \) with similar-looking characters based on a keyboard layout. This subtle manipulation aims to create a new input \( x^* \) that retains readability while causing the model to misclassify the text.

This variant introduces a gradient-free optimization approach, replacing the traditional iterative update rule with a gradient-free strategy inspired by the gradient-free optimization technique used in M-PruthiKeyabord. This modification allows for more efficient and scalable adversarial example generation.