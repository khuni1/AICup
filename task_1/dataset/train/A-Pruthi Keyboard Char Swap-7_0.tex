%Input
$\mathbf{x}_{\text{original}}$: Original text input sequence.
$f(\mathbf{x})$: Target model (e.g., text classifier).
$y_{\text{true}}$: True label of the input sequence.
$\epsilon$: Maximum allowed number of character swaps.
$\mathcal{N}(c)$: Set of neighboring characters for a character $c$, based on a keyboard layout.

%Output
Adversarial example $\mathbf{x}_{\text{adv}}$ such that:
\[
f(\mathbf{x}_{\text{adv}}) \neq y_{\text{true}}, \quad \text{with minimal changes}.
\]

%Formula
1. Initialization:
   \[
   \mathbf{x}_{\text{adv}}^{(0)} = \mathbf{x}_{\text{original}}.
   \]

2. Iterative Character Substitution:
   - For each character $c_i$ in $\mathbf{x}_{\text{adv}}$:
     \[
     c_i' = \underset{c' \in \mathcal{N}(c_i)}{\arg \max} \, \mathcal{L}(f, \mathbf{x}_{\text{adv}}^{(i \rightarrow c')}, y_{\text{true}}),
     \]
     where $\mathbf{x}_{\text{adv}}^{(i \rightarrow c')}$ represents the input with $c_i$ replaced by a neighboring character $c'$.

3. Update the Input:
   \[
   \mathbf{x}_{\text{adv}}^{(t+1)} = \mathbf{x}_{\text{adv}}^{(t)} \, \text{with the replaced character}.
   \]

4. Stopping Condition:
   - If $f(\mathbf{x}_{\text{adv}}^{(t+1)}) \neq y_{\text{true}}$, or if the number of modified characters exceeds $\epsilon$, terminate the attack.

%Explanation
Pruthi's Keyboard Character Swap Attack is a heuristic-based adversarial attack designed for natural language processing (NLP) tasks.

1. Objective: The attack generates adversarial text by swapping characters in a way that mimics real-world typing errors (e.g., adjacent key swaps), causing misclassification by the target model.
2. Keyboard Layout: The attack uses a predefined keyboard layout to identify valid neighboring characters for each character in the input.
3. Iterative Substitution: Characters are iteratively replaced with their neighboring counterparts, prioritizing those changes that increase the model's loss the most.
4. Semantic Preservation: The character swaps are designed to resemble realistic typographical errors, maintaining the overall readability and semantic meaning of the text.
5. Non-Gradient Based: This attack does not require access to the model's gradients, making it applicable even in scenarios where the target model is a black box.