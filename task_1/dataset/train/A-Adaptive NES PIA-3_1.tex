%Input
\mathbf{x}: Original input instance.  
$f_{\theta}(\mathbf{x})$: Target model with parameters $\theta$.  
y: True label of the input instance.  
$\mathbf{p}^{(0)}$: Initial perturbation sampled from $\mathcal{N}(0, I)$.  
$\Delta$: Accumulated perturbation update.  
N: Population size for NES optimization.  
$\sigma$: Mutation step size.  
$N_{\text{iter}}$: Number of iterations for the optimization process.  
$L_{\text{attack}}(\mathbf{x}_{\text{adv}}, y)$: Adversarial loss function with a penalty term.  
$\lambda$: Scaling factor for the final perturbation.

%Output
$\delta_{\text{opt}} = \frac{\Delta^{(N_{\text{iter}})}}{\|\Delta^{(N_{\text{iter}})}\|_2}$

%Formula
\begin{itemize}
    \item Initialize the perturbation parameter $\mathbf{p}^{(0)}$ as a random perturbation sampled from a normal distribution, e.g., $\mathbf{p}^{(0)} \sim \mathcal{N}(0, I)$.
    \item Set $\Delta = 0$ to accumulate the perturbation changes across iterations.
    \item Define the population size $N$ (number of candidate perturbations) and the learning rate $\sigma$ (for mutation).
    \item Define the number of iterations $N_{\text{iter}}$ for the optimization process.
\end{itemize}

For each iteration $t = 1, \ldots, N_{\text{iter}}:$
\begin{itemize}
    \item Perturbation Evolution: $\mathbf{p}^{(t)} = \mathbf{p}^{(t-1)} + \sigma \cdot \mathcal{N}(0, I)$.
    
    \item Adversarial Loss Computation: $L_{\text{attack}}^{(t)} = -\frac{\|\mathbf{x}_{\text{adv}} - y\|_2^2}{P_{\text{max}}}$.

    \item Update Perturbation: $\mathbf{p}^{(t)} \leftarrow \mathbf{p}^{(t)} - \alpha \cdot \nabla_{\mathbf{p}} L_{\text{attack}}^{(t)}$.
    
    \item Update Perturbation Accumulation: $\Delta^{(t)} \leftarrow \Delta^{(t-1)} + \mathbf{p}^{(t)}$.
\end{itemize}

After the final iteration $t = N_{\text{iter}}:$
Final Perturbation: $\delta_{\text{opt}} = \frac{\Delta^{(N_{\text{iter}})}}{\|\Delta^{(N_{\text{iter}})}\|_2}$.
    
Adversarial Example: $\mathbf{x}_{\text{adv}} = \mathbf{x} + \lambda \delta_{\text{opt}}$.

%Explanation
The NES PIA attack is an optimization-based adversarial attack that uses Natural Evolution Strategies (NES) to generate adversarial perturbations. The attack can be adjusted by changing the number of iterations $N_{\text{iter}}$, the mutation step size $\sigma$, and the scaling factor $\lambda$. 

The proposed variant modifies the NES PIA attack to incorporate a new constraint, which is a modified loss function that incorporates an additional penalty term. This modification aims to improve the robustness of the attack against certain types of defenses.

In comparison to the original NES PIA attack, the modified version has two main differences:

1.  The new loss function incorporates an additional penalty term, which makes the optimization process more challenging and potentially more effective.
2.  The mutation step size $\sigma$ is now adjusted based on the magnitude of the perturbation accumulation $\Delta$, which can help to improve the stability and efficiency of the attack.

These modifications aim to improve the robustness and effectiveness of the NES PIA attack, making it a stronger tool for adversarial attacks.