%Input
The input to the system is a real adversarial image $\tilde{v}_{adv}$ and a corresponding integer image $\tilde{d}$ that needs to be denormalized.

%Output
The output of the denormalization and classification process involves:
1. The denormalization relationship: $T^{-1}(T(\tilde{d})) = \tilde{d} \quad \forall \tilde{d} \in D$.
2. The classification after denormalization: $f_t(T^{-1}(\tilde{v}_{adv})) = g_t(T(T^{-1}(\tilde{v}_{adv}))) \neq g_t(\tilde{v}_{adv})$.
3. The discretization error: $\text{Discretization Error} : g_t(T(T^{-1}(\tilde{v}_{adv}))) \neq g_t(\tilde{v}_{adv})$.
4. A potential misclassification: $f_t(T^{-1}(\tilde{v})) \neq c$.


%Formula
Denormalization Relationship
$T^{-1}(T(\tilde{d})) = \tilde{d} \quad \forall \tilde{d} \in D$

Discretization Problem
$f_t(T^{-1}(\tilde{v}_{adv})) = g_t(T(T^{-1}(\tilde{v}_{adv}))) \neq g_t(\tilde{v}_{adv})$

Discretization Error Definition
$\text{Discretization Error} : g_t(T(T^{-1}(\tilde{v}_{adv}))) \neq g_t(\tilde{v}_{adv})$

Classification After Transformation
$f_t(T^{-1}(\tilde{v})) \neq c$


%Explanation
Discretized Inputs Denormalization Discretization Error where:
- The denormalization function $T^{-1}$ is designed to convert a real-valued image back into its corresponding discrete integer representation.
- However, this transformation can lead to discrepancies between the original class prediction $g_t(\tilde{v}_{adv})$ and the class prediction after transformation, represented as $f_t(T^{-1}(\tilde{v}_{adv}))$.
- This mismatch is referred to as the discretization problem, which can affect the efficacy of adversarial attacks by resulting in misclassifications.