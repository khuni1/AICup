%Input
Original input $x$ and true label $y_{\text{true}}$.

Perturbation parameters:
- Maximum perturbation size $\epsilon$
- Number of iterations $N$
- Lagrange multiplier $\lambda$
- Spectral constraint function $S(x)$

Model function: $f_{\theta}(x)$ with loss function $L(f_{\theta}(x), y)$.

%Output
The output is an adversarial example $x^*$ that is crafted to deceive the model while adhering to specific constraints in the spectral domain.

%Formula
1. Initialize the input:
   $x^{(0)} = x$
2. Set parameters for the optimization process:
   - Define the maximum perturbation size $\epsilon$, the number of iterations $N$, and the Lagrange multiplier $\lambda$.
3. For each iteration $n = 1$ to $N$:
   - Compute the adversarial perturbation:
   $\delta_n = -\alpha \cdot \nabla_x L(f_{\theta}(x^{(n-1)}), y)$
   - Update the input while considering spectral domain constraints:
   $x^{(n)} = x^{(n-1)} + \delta_n + \lambda \cdot S(x^{(n-1)})$
   where $S(x)$ is a function that enforces spectral domain constraints.
   - Apply clipping to ensure the perturbation stays within bounds:
   $x^{(n)} = \text{Clip}_{\mathcal{X}}(x^{(n)})$
   ensuring:
   $\|x^{(n)} - x\|_p \leq \epsilon$

4. The final adversarial example is:
   $x^* = x^{(N)}$

%Explanation
The Spectral-Constrained Adv-Lagrangian Attack (SCALA) adversarial attack aims to generate adversarial examples that effectively mislead a target model while respecting constraints related to the spectral domain. By incorporating a Lagrange multiplier, the attack balances the adversarial objective with adherence to spectral fidelity. This approach not only seeks to alter the model's prediction but also ensures that the generated inputs remain perceptually similar to the original images in terms of spectral distribution. The resulting adversarial example $x^*$ underscores the importance of considering perceptual constraints in adversarial machine learning, particularly in applications involving image classification.

Summary: This variant differs from the main ReColorAdv-Lagrangian attack by incorporating a new constraint related to the spectral domain, which is enforced through a separate function S(x). This modification allows the attack to target specific spectral features of images while maintaining the core principle of the original attack.