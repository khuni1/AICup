%Input
Let $x$ be the original input image, and $f_{\theta}$ be the target model. The objective is to create an adversarial example $x^*$ that misleads the model while optimizing for a custom similarity metric.

%Output
The output of the Cosine Similarity attack with Custom Scoring is an adversarial example $x^*$ that is generated by maximizing a custom scoring function that combines cosine similarity and Euclidean distance.

%Formula
1. Initialize the input:
$x^{(0)} = x$
2. Set parameters for the optimization process:
	* Define the maximum perturbation size $\epsilon$ and the learning rate $\alpha$.
3. For each iteration $n = 1$ to $N$:
	* Compute the embedding:
    $e_n = f_{\theta}(x^{(n-1)})$
    where $e_n$ is the embedding of the current input.
	* Calculate the gradient with respect to a custom scoring function:
    $g_n = \nabla_x (c_1 \cdot (1 - \frac{e_n \cdot e_t}{\|e_n\| \|e_t\|}) + c_2 \cdot (\|x^{(n-1)} - x\|_p)^p)$
    where $c_1$ and $c_2$ are hyperparameters controlling the scoring function.
	* Update the input:
    $x^{(n)} = x^{(n-1)} - \alpha \cdot g_n$
4. The final adversarial example is:
    $x^* = x^{(N)}$

%Explanation
The Cosine Similarity attack with Custom Scoring modifies the original Embedding Cosine attack by introducing a custom scoring function that combines cosine similarity and Euclidean distance. This scoring function allows for more control over the type of perturbation generated, enabling the creation of adversarial examples that are tailored to specific objectives or constraints. The custom scoring function can be adjusted through hyperparameters $c_1$ and $c_2$, allowing for trade-offs between maximizing cosine similarity and minimizing Euclidean distance. This Custom-Scored Cosine Embedding Attack (CS-CEA) variant maintains the core principle of the original attack while introducing a new twist that enables more targeted and effective adversarial examples.