%Input
\textbf{Input:}  
\[
x \in \mathbb{R}^d, \quad f: \mathbb{R}^d \to \mathbb{R}
\]

where:  
- \( x \) is the input sample (e.g., an image),  
- \( f(x) \) is the classifier function that maps an input to a real-valued decision score.

%Output
The DeepFool with Arctanh-Space Transformation generates a perturbation \( \hat{r} \) to misclassify an image \( x \). Unlike the standard DeepFool attack, this variant operates in the transformed \(\tanh\)-space to enforce bounded perturbations while still minimizing the L2 distance.

%Formula
\[
\hat{r} = -\frac{f(\tanh(x_i))}{\|\nabla f(\tanh(x_i))\|_2^2} \nabla f(\tanh(x_i)).
\]

%Explanation
The DeepFool with Arctanh-Space Transformation introduces a transformation step where the input \( x \) is projected into the \(\tanh\)-space, ensuring that the generated perturbation remains within a bounded range. This helps prevent large distortions while still achieving misclassification. The attack proceeds iteratively:

1. Transform the input using the inverse hyperbolic tangent function:
   \[
   \tilde{x}_0 = \text{arctanh}(x)
   \]
2. Compute the gradient-based perturbation:
   \[
   r_i = -\frac{f(\tanh(x_i))}{\|\nabla f(\tanh(x_i))\|_2^2} \nabla f(\tanh(x_i)).
   \]
3. Update the perturbed image in the transformed space:
   \[
   \tilde{x}_{i+1} = \tilde{x}_i + r_i.
   \]
4. Transform back to the original space:
   \[
   x_{i+1} = \tanh(\tilde{x}_{i+1}).
   \]
5. Repeat until the classifier prediction changes.

The final perturbation \( \hat{r} \) is the cumulative sum of all step-wise perturbations, ensuring minimal L2 distortion while achieving adversarial misclassification.
