%Input
$X$: The set of data points. 
$\hat{k}$: The classifier function. 
$\xi$: The desired $\ell_p$ norm of the perturbation. 
$\delta$: The desired accuracy on perturbed samples. 
$v$: The universal perturbation vector. 
$\text{Err}(Xv)$: The error rate on the perturbed data points $Xv$. 
$x_i$: A data point in $X$. 
$\Delta v_i$: The minimal perturbation required to change the classification of $x_i + v$. 
$\| \cdot \|_2$: The $\ell_2$ norm. 
$P_{p, \xi}$: The projection operator that projects $v$ onto the $\ell_p$ ball of radius $\xi$.

%Output
Universal perturbation vector $v$

%Formula
Computation of Universal Perturbations:

1. Initialize the perturbation vector $v \leftarrow 0$.

2. While the error rate on the perturbed data is less than or equal to $1 - \delta$:
\[
\text{Err}(Xv) \leq 1 - \delta
\]

3. For each data point $x_i \in X$, do:
    - If the classifier's prediction for $x_i + v$ equals the prediction for $x_i$:
    \[
    \hat{k}(x_i + v) = \hat{k}(x_i)
    \]
        - Compute the minimal perturbation $\Delta v_i$ that sends $x_i + v$ to the decision boundary:
        \[
        \Delta v_i \leftarrow \arg \min_{r} \|r\|_2 \quad \text{s.t.} \quad \hat{k}(x_i + v + r) \neq \hat{k}(x_i)
        \]
        - Update the perturbation:
        \[
        v \leftarrow P_{p, \xi}(v + \Delta v_i)
        \]

4. Repeat the process until the error rate condition is no longer met.

Return the final universal perturbation vector is:
\[
v
\]

%Explanation
Universal Adversarial Perturbation (UAP) can be summarised as:
1. Initialization: 
Initialize the universal perturbation vector $v$ to 0.

2. Iteration Loop: 
Continue iterating while the error rate on the perturbed data points $\text{Err}(Xv)$ is less than or equal to $1 - \delta$.

3. Perturbation Update for Each Data Point: 
For each data point $x_i$ in the set $X$: 
If the classifier's prediction for $x_i + v$ is the same as for $x_i$: 
Compute the minimal perturbation $\Delta v_i$ required to change the classification of $x_i + v$. This is done by finding the smallest perturbation $r$ such that the classifier's prediction for $x_i + v + r$ differs from its prediction for $x_i$. 
Update the perturbation $v$ by adding $\Delta v_i$ and then projecting the result onto the $\ell_p$ ball of radius $\xi$ using the projection operator $P_{p, \xi}$.

4. Return the universal perturbation vector $v$.
