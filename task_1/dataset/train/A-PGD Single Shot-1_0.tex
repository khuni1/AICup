%Input
$x' = \text{clip}(x + \epsilon \cdot \text{sign}(\nabla_x J(\theta, x, y)), x - \epsilon, x + \epsilon)$

% Output
The perturbed data point $x'$ is the produced adversary sample.

%Formula
$x' = \text{clip}(x + \epsilon \cdot \text{sign}(\nabla_x J(\theta, x, y)), x - \epsilon, x + \epsilon)$

Single Shot Perturbation, Compute the gradient of the loss function:
\[
g = \nabla_x J(\theta, x, y)
\]

Calculate the perturbed data point:
\[
x' = \text{clip}(x + \epsilon \cdot \text{sign}(g), x - \epsilon, x + \epsilon)
\]

\subsection*{Return}
The final perturbed input is:
\[
x'
\]

%Explanation
Project Gradient Descent (PGD) single Shot:
$x$ is the original input data point,
$x'$ is the perturbed data point,
$\epsilon$ is the perturbation strength,
$\nabla_x J(\theta, x, y)$ is the gradient of the loss function $J$ with respect to the input data $x$,
$\theta$ represents the parameters of the model being attacked.
The \texttt{clip} function ensures that the perturbed data point remains within a specified range around the original data point.