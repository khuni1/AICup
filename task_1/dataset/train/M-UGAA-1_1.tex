%Input
Let \( x_i \) be the original input data points from a dataset \( D \), \( y_i \) be the true labels, and \( \epsilon \) be the maximum allowable perturbation. The UAP Attack aims to generate a universal perturbation \( \delta \) that can deceive the model across a variety of inputs.

%Output
The output is a universal perturbation $\delta$ that can be added to various inputs to mislead the classifier.


%Formula
1. Initialize the universal perturbation \( \delta \):
   $
   \delta = 0.
   $
2. Define the objective function to minimize the loss for all inputs:
   $
   \text{minimize } \frac{1}{N} \sum_{i=1}^N L(f(x_i + \delta), y_i),
   $
   where \( f \) is the model and \( L \) is the loss function.
3. Update the perturbation using gradient descent:
   $
   \delta^{(t+1)} = \delta^{(t)} + \alpha \cdot \text{sign} \left( \frac{1}{N} \sum_{i=1}^N \nabla_{\delta} L(f(x_i + \delta^{(t)}), y_i) \right),
   $
   where \( \alpha \) is the step size and \( N \) is the number of samples.
4. Project \( \delta \) to ensure it remains within the allowable perturbation bounds:
   $
   \delta = \text{clip}(\delta, -\epsilon, \epsilon).
   $

%Explanation
The Universal Gradient Aggregation Attack (UGAA) variant generates a universal adversarial perturbation \( \delta \) that can be applied to multiple inputs \( x_i \) in order to mislead the model. By optimizing \( \delta \) to minimize the average loss across a set of inputs, the attack effectively creates a perturbation that generalizes well to various data points. This approach demonstrates the potential for a single perturbation to compromise the classification of a range of inputs, highlighting a significant vulnerability in machine learning models to universal adversarial attacks.